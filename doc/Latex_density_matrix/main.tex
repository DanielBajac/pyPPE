%%%%%%%%%%%%%%%%%%%%%%%%%%%%%%%%%%%%%%%%%
% fphw Assignment
% LaTeX Template
% Version 1.0 (27/04/2019)
%
% This template originates from:
% https://www.LaTeXTemplates.com
%
% Authors:
% Class by Felipe Portales-Oliva (f.portales.oliva@gmail.com) with template 
% content and modifications by Vel (vel@LaTeXTemplates.com)
%
% Template (this file) License:
% CC BY-NC-SA 3.0 (http://creativecommons.org/licenses/by-nc-sa/3.0/)
%
%%%%%%%%%%%%%%%%%%%%%%%%%%%%%%%%%%%%%%%%%

%----------------------------------------------------------------------------------------
%	PACKAGES AND OTHER DOCUMENT CONFIGURATIONS
%----------------------------------------------------------------------------------------

\documentclass[
	12pt, % Default font size, values between 10pt-12pt are allowed
	%letterpaper, % Uncomment for US letter paper size
	%spanish, % Uncomment for Spanish
]{fphw}

% Template-specific packages
\usepackage[utf8]{inputenc} % Required for inputting international characters
\usepackage[T1]{fontenc} % Output font encoding for international characters
\usepackage{mathpazo} % Use the Palatino font

\usepackage{graphicx} % Required for including images

\usepackage{booktabs} % Required for better horizontal rules in tables

\usepackage{listings} % Required for insertion of code
\usepackage[utf8]{inputenc}
\usepackage{amsmath}
\usepackage{amsfonts}
\usepackage{amsthm}
\usepackage{amssymb}
\usepackage{mathrsfs}
\usepackage{enumitem}
\usepackage{physics}
\usepackage{bm}
\usepackage{enumerate} % To modify the enumerate environment
\usepackage{hyperref}
\usepackage[backend=biber,style=numeric,sorting=none]{biblatex}
\addbibresource{refs.bib}
%----------------------------------------------------------------------------------------
%	ASSIGNMENT INFORMATION
%----------------------------------------------------------------------------------------

\title{Matriz densidadd Reeducida} % Assignment title

\author{Daniel F. E. Bajac} % Student name

\date{9 de diciembre, 2021} % Due date

\institute{Universidad Nacional del Nordeste \\ Dep artamento de Física} % Institute or school name

\class{Propagadores de Polarización} % Course or class name

\professor{Dr. G. A. Aucar y A. F. Maldonado} % Professor or teacher in charge of the assignment

%----------------------------------------------------------------------------------------

\begin{document}

\maketitle % Output the assignment title, created automatically using the information in the custom commands above

%----------------------------------------------------------------------------------------
%	ASSIGNMENT CONTENT
%----------------------------------------------------------------------------------------

\section*{Definición del problema}

En el año 2014, se definió por primera vez una funcional generatriz para el formalismo
de Propagadores de Polarización utilizando formalismo de superoperadores\cite{QFT}. Ésta funcional 
está escrita en función del Propagador Principal, uno de los principales componentes del Propagador
de Polarización, que contiene toda la información física que surge debido a la transmisión
de los efectos de dos perturbaciones externas a través del marco electrónico del sistema cuántico

\begin{equation}\label{Z}
	Z = \int D| \bm{\widetilde{h}} ) D (\bm{h}|  e^{| \bm{\widetilde{h}} ) (\bm{h}| E \hat{I} - \hat{H}_0 | \bm{\widetilde{h}} )   (\bm{h}|}
\end{equation}

donde $E$ es la energía del sistema y $H_0$ es el Hamiltoniano del sistema sin perturbar, $\bm{h}$ es un conjunto de operadores 
completo con el cual se puede generar todos los estados excitados de un sistema molecular N-electrónico, 
$\bm{h} | \bm{0} \rangle = |\bm{n}\rangle$  y el estado de referencia
es un campo autoconsistente SCF.




En la función de partición \ref{Z} 
la integral significa que la exponencial tomará todos los caminos posibles, lo cual se representa con una sumatoria 
sobre todos los caminos posibles, es decir, los ${i,j}$ pueden tomar el lugar de cualquier estado ocupado del sistema, 
y los ${a,b}$, el lugar de cualquier estado desocupado   

\begin{equation}\label{Z_sum}
	Z = \sum_{ia,jb}  e^{| \bm{\widetilde{h}}_2 ) (\bm{h}_2| E \hat{I} - \hat{H}_0 | \bm{\widetilde{h}}_2 )   (\bm{h}_2|}
\end{equation}
Ésta función generatriz es análiga a la función de partición de la termodinámica estadística \cite{functional} 
($Z = Tr (e^{\beta \hat{H}})$),con la cual el operador densidad  del sistema se puede escribir


\begin{equation}\label{rho}
	\rho = \frac{e^{| \bm{\widetilde{h}} ) (\bm{h}| E \hat{I} - \hat{H}_0 | \bm{\widetilde{h}} )   (\bm{h}|}}
	{Z}
\end{equation}


 
En el formalismo de superoperadores, es importante definir el conjunto de operadores completo:

\begin{equation}
	\bm{h} =    \left\{ \bm{h}_2, \bm{h}_4, \cdots \right\} 
\end{equation}

A los fines del presente trabajo, se considera tratar la matriz densidad a nivel de teoría RPA, 
que implica considerar al estado de referencia un SCF 
y truncar el conjunto de operadores en excitaciones simples \cite{Revieww} $ \bm{h} = \bm{h}_2 $ 


\begin{equation}
	\bm{h}_2 = \left\{a^\dagger_a a_i, a^\dagger_i a_a, \cdots\right\}
\end{equation}

donde $a^\dagger_i$ es un operador creación en el orbital ocupado $i$ y 
$a_a$ es un operador aniquilación en el orbitales virtual $a$. 
Considerar el siguiente nivel del conjunto de operadores significa considerar excitaciones dobles $\bm{h}_4$, 
además de utilizar un estado de referencia con estados doblemente excitados.
  




Los operadores $ \bm{h}_2 $ y $ \bm{\widetilde{h}}_2 $ son operadores 
fila y columna respectivamente. Además, el producto interno entre dos operadores está definido como 

\begin{equation}
	(\hat{P}|\hat{Q}) = \langle 0 | [\hat{P}^\dagger ,\hat{Q}] |0 \rangle
\end{equation}


\textit{Se desea verificar la matriz densidad reducida a ciertas excitaciones, por ejemplo, $i\rightarrow a,j\rightarrow b$.}
Es decir, obtener el operador densidad que corresponde a dichas excitaciones, $\rho_{ia,jb}$, y que previamente se ha utilizado
para conocer cómo están entrelazadas cuánticamente dichas excitaciones, aplicándole Teoría de la Información .\cite{Millan}
 \cite{JCP}.

 \begin{equation*}
	 \rho_{ia,jb} =  \frac{e^{| \bm{\widetilde{h}}_{ia} ) (\bm{h}_{ia}| E \hat{I} - \hat{H}_0 | \bm{\widetilde{h}}_{jb} )   (\bm{h}_jb|}}
	 {Z_{ia,jb}}
 \end{equation*}


\section*{Resolución de la traza parcial de $\rho$}

Dado $\hat{O}$ un operador lineal definido en el espacio de Hilbert $H$, en la mecánica cuántica de distemas compuestos 
con el espacio de Hilbert $H = H_a \otimes H_b$, se define la función \textit{función
traza parcial}, tomado sobre el subsistema \textit{b}, como \cite{RDM}

\begin{equation}\label{Tr_b}
	Tr_b(\hat{O}) = \sum_{j=1}^{d_b} (\bm{1_a} \otimes \langle b_j|) \hat{O} (\bm{1_a} \otimes |b_j \rangle)
\end{equation}

donde $|b_j\rangle$ cualquier base ortonormal para $H_b$ y $d_b= dim(H_b)$. The $\bm{1_a}$ es el operador identidad en $H_a$.

Es importante notar que la anterior definición es equivalente a otra, que aparece muy frecuentemente en la literatura\cite{RDM} \cite{Nielsen_Chuang}

\begin{equation}\label{Tr_b_2}
	Tr_b( |a \rangle \langle a' | \otimes | b\rangle \langle b' | ) = |a \rangle \langle a' | \otimes Tr (|b\rangle \langle b' |)
\end{equation}

donde $|a \rangle$, $|a' \rangle$ $\epsilon$ $H_a$ y $|b\rangle$, $|b'\rangle$ $\epsilon$ $H_b$ son vectores genéricos en 
los correspondientes espacios de Hilbert.

Se puede utilizar ambas definiciones para definir una función de traza parcial para la matriz densidad utilizando 
el formalismo de Propagadores de Polarización \ref{rho}. 
Si consideramos a los elementos de $\rho$ correspondentes a dichas excitaciones ($i\rightarrow a, j \rightarrow b)$ como un subsistema 
(subsistema $A$), 
y a los elementos de $\rho$ correspondientes al resto de las excitaciones como otro subsistema (subsistema $B$), podemos
hallar la matriz densidad reducida $\rho_{ia,jb}$ haciendo la traza de $\rho$ con respecto al sub-sistema $B$. %\cite{Nielsen_Chuang}
Para eso, definimos al operador identidad del espacio de 
Fock que corresponde a las excitaciones $i\rightarrow a,j\rightarrow b$ como 

\begin{equation}
	\bm{1}_{ia,jb} = \bm{1}_{ia} \otimes \bm{1}_{jb} = \left\{  \bm{1}_{ia} ,  \bm{1}_{jb} \right\}
\end{equation}

lo cual significa que en las posiciones de las excitaciones $ia,jb$ los elementos matriciales de la matriz densidad
quedan invariantes.


Por otro lado, se propone que la traza sobre el algún sistema se obtiene multiplicando a izquierda por 
los operadores $\bm{\hslash}_{p,q}$, con dimensión igual a $\bm{h_2}$, con un operador \textit{hueco-partícula} y 
\textit{partícula-hueco} en el lugar de excitación $p\rightarrow q$ y cero en las restantes, y 
sumamos sobre todos los $p,q$

\begin{equation}\label{Tr_rho_total}
	Tr(\rho) =  \sum_{p,q} ( \bm{\hslash}_{p,q} | \rho  \bm{\widetilde{\hslash}_{p,q}} )   
\end{equation}

\begin{equation}
	( \bm{\hslash}_{p,q} | = \left(\begin{matrix}
		0 \\
		\cdots \\
		(h_{pq}| \\
		\cdots \\
		0
	\end{matrix} \right)
\end{equation}

Y a derecha por operadores filas del tipo

\begin{equation}
	| \widetilde{\bm{\hslash}}_{p,q} ) = \begin{matrix}
		( 0 &\cdots& |h_{p,q})& \cdots& 0)
	\end{matrix}
\end{equation}

donde definimos a cada operador $h_{p,q}$ como un conjunto de operadores 
\textit{partícula-hueco} y \textit{hueco-partícula} entre los orbitales ocupados y desocupados $p,q$.

\begin{equation}
	h_{p,q} = {a^\dagger_p a_q, a^\dagger_p a_q}
\end{equation}

Por lo que, la traza parcial sobre el sistema que llamamos $b$ es 

\begin{equation}
	Tr_B(\bm{\rho}) = \sum_{p,q}  \left(\bm{1}_a \otimes ( \bm{\hslash}_{p,q} | \right) \bm{\rho}  \left( \bm{1}_a \otimes |\bm{\hslash}_{p,q}) \right)
\end{equation}

También, si nos basamos en la definición \ref{Tr_b_2}, podemos escribir la traza parcial sobre el sistema $B$ como 

\begin{equation}
	Tr_B(\bm{\rho}) = \rho_a Tr (\rho_b)
\end{equation}

donde ésta última ecuación es el esquema que adoptaremos de aquí en adelante.

\subsection*{Numerador}


%\begin{equation}
%	\rho_{ia,jb} = Tr_B (\rho) 
%\end{equation}

%Sabiendo cómo se calcula la traza de una matriz densidad en el formalismo de funciones de onda,

%\begin{equation}
%	Tr(e^{-\beta \hat{H}}) = \sum_n \langle \phi_n | e^{-\beta \hat{H}} | \phi_n \rangle
%\end{equation}

%se calcula la traza en el formalismo de Propagadores de Polarización 

%Sin embargo, como se necesita calcular la traza sobre los elementos de B, debemos dejar \textit{invariantes} a los elementos de
%$p,q$ pertenecientes a las excitaciones $i\rightarrow a,j\rightarrow b$ . Por lo que, en los operadores 
%$\bm{\hslash}$ contendrán un operador identidad en las posiciones $i,a$ y $j,b$ , como se muestra a continuación

%\begin{equation}\label{Tr_rho}%
%	Tr_B (\rho) =  \sum_p (( \bm{\hslash}_{p,q} | \rho  \widetilde{\bm{\hslash}}_{p,q} )   
%\end{equation}


%\begin{equation}\label{Tr_reducida}
%	( \bm{\hslash}_{p,q} | = \left(\begin{matrix}
%		0 \\
%		\hat{I}_{(i,a)} \\
%		0 \\
%		\hat{I}_{(j,b)} \\
%		\cdots \\
%		(h_{pq}| \\
%		\cdots \\
%		0
%	\end{matrix} \right)
%\end{equation}

%y de manera similar para el operador fila $\widetilde{\bm{\hslash}}_p$.  
%Esto nos permite escribir la matriz densidad como el producto de dos matrices densidad

sabiendo que el sistema N-electrónico puede escribirse 

\begin{eqnarray}
	\rho &=& \rho_{ia_jb} \otimes \rho_{B} \\
	&=& e^{| {\bm{\widetilde{h}}_{ia,jb} ) (\bm{h}_{ia,jb}| E \hat{I} - \hat{H}_0 | 
	\bm{\widetilde{h}}_{ia,jb} )   (\bm{h}_{ia,jb}|}} \otimes
	e^{| {\bm{\widetilde{h}}_B ) (\bm{h}_B| E \hat{I} - \hat{H}_0 | 
	\bm{\widetilde{h}}_B )   (\bm{h}_B|}} \times \frac{1}{Z}
\end{eqnarray}

%Por otro lado, 
y sabiendo que se puede escribir la exponencial de un operador como una serie de potencias del operador

\begin{equation}\label{exp_rho}
	e^{\hat{x}} = \hat{I} + \hat{x} + \frac{\hat{x}^2}{2} + \cdots
\end{equation}



\begin{eqnarray}\label{serie_rho}
	Tr_B (\rho) &=&  \rho_{ia,jb}  \sum_{p,q} ( \bm{\hslash}_{p,q} | \hat{I} + \frac{ | \bm{\widetilde{h}}_2 ) (\bm{h}_2| E \hat{I} - \hat{H}_0 |
	 \bm{\widetilde{h}}_2 )   (\bm{h}_2|  + \cdots}{Z_{[0]}}  \widetilde{\bm{\hslash}}_{p,q} ) \\
	 &=& \rho_{ia,jb} \sum_{p,q} \left[ ( \bm{\hslash}_{p,q} |  \hat{I} \widetilde{\bm{\hslash}}_{p,q} ) 
	  + ( \bm{\hslash}_{p,q} | \frac{(\bm{h}_2| E \hat{I} - \hat{H}_0 |
	 \bm{\widetilde{h}}_2 )   (\bm{h}_2|}{Z_{[0]}} \widetilde{\bm{\hslash}}_{p,q} ) + \cdots \right]   
\end{eqnarray}






\subsubsection*{Términos de la expansión en serie de la exponencial}
Primero, se evalúa el producto $( \bm{\hslash}_{p,q} |\hat{I} \bm{\widetilde{\hslash}}_{p,q} )$


\begin{eqnarray}
	( \bm{\hslash}_{p,q} |\hat{I} \bm{\widetilde{\hslash}}_{p,q} ) &=& 
	\begin{pmatrix}
		0 \\\cdots\\
		(h_{pq}| \\
		\cdots \\
		0 \end{pmatrix}
 \times \begin{pmatrix}
	0 & \cdots & |h_{p,q})& \cdots& 0
\end{pmatrix} \\
	&=& \begin{pmatrix}
		0 & \cdots &  \cdots  \\
		\vdots & (h_{pq} | h_{pq}) & \vdots \\
		\vdots & 0 & \cdots
		\end{pmatrix}
\end{eqnarray}

por lo que obtendremos una matriz con todos sus elementos cero, menos en el elemento $p,q$, 
con un valor $\langle0| [h^\dagger_{p,q}, h_{p,q}] |0\rangle$. 
Como hemos definido, los elementos $p,q$ de $h_{p,q}$ son los orbitales ocupados y desocupados, 
diferentes de $i,j$ y $a,b$.
El producto interno $(h_{pq} | h_{pq})$ está definido como 

\begin{eqnarray}
	(h_{pq} | h_{pq}) &=& \begin{pmatrix}
		(a^\dagger_p a_q| a^\dagger_p a_q) &  (a^\dagger_p a_q| a^\dagger_q a_p) \\
		(a^\dagger_q a_p| a^\dagger_p a_q)  & (a^\dagger_q a_p| a^\dagger_q a_p)
	\end{pmatrix} 
	= \begin{pmatrix}
		1 &  0 \\
		0 &  -1
		\end{pmatrix}
\end{eqnarray}

%\begin{equation}%
%	\sum_{p,q} ( \bm{\hslash}_{p,q} |\hat{I} \bm{\widetilde{\hslash}}_(p,q) ) = \sum_{p,q} $\langle0| [h^\dagger_{p,q}, h_{p,q}] |0\rangle$
%\end{equation}


Por lo que hacer la sumatoria sobre todos los orbitales $p,q$ perteneciantes al subsistema  
$\sum_{p,q} (\bm{\hslash}_{p,q} |\hat{I} \bm{\widetilde{\hslash}}_{p,q})$ es igual a 0.


El siguiente paso es analizar los elementos matriciales del término de primer órden de la ecuación \ref{serie_rho}-
El término $( \bm{\hslash}_{p,q} | \bm{\widetilde{h}}_2 )$ está compuesto por

\begin{eqnarray}
	( \bm{\hslash}_{p,q} | \bm{\widetilde{h}}_2 ) &=& \begin{pmatrix}
		
		0 \\
		\vdots \\
		(h_{pq}| \\
		\vdots \\
		0
	\end{pmatrix}
	\times \begin{pmatrix}
		
		 |h_{i'a'}) & |h_{j',b'}) & \cdots  & \cdots
	\end{pmatrix} \end{eqnarray}

por ejemplo, para $( \bm{\hslash}_{i',j'} | $, tenemos 

\begin{eqnarray} \label{hslash_qp}
	( \bm{\hslash}_{p,q} | \bm{\widetilde{h}}_2 )
		 &=& \begin{pmatrix}
		(h_{i',a'}) \\
		\vdots \\
		0 \\
		\vdots \\
		0
	\end{pmatrix}  \times  \begin{pmatrix}
		|h_{i'a'}) & \cdots & |h_{j',b'}) &  \cdots
	\end{pmatrix}            \\
     & = & \begin{pmatrix}
		 (h_{i',a'} | h_{i',a'}) & 0 & \cdots \\
		 0 & \vdots & \vdots \\
		 \vdots & \vdots & \vdots 
	 \end{pmatrix} 
\end{eqnarray}

mientras que a derechas, tenemos que resolver el producto de matrices 
$( \bm{h}_2 | \widetilde{\bm{\hslash}}_{p,q})$

\begin{eqnarray}\label{hslash_pq}
	( \bm{\hslash}_{p,q} | \bm{\widetilde{h}}_2 )
		 &=& \begin{pmatrix}
		(h_{i',a'}| \\
		 \vdots \\
		(h_{j',b'}| \\
		
		\vdots 
	\end{pmatrix}  \times  \begin{pmatrix}
		|h_{i'a'}) & 0 &  \cdots
	\end{pmatrix}            \\
     & = & \begin{pmatrix}
		 (h_{i',a'} | h_{i',a'}) & 0 & \cdots \\
		 0 & \vdots & \vdots \\
		 \vdots & \vdots & \vdots 
	 \end{pmatrix} 
\end{eqnarray}

y a la matriz central,$(\bm{h}_B| E \hat{I} - \hat{H}_0 | \bm{\widetilde{h}}_B )$ 
que es el la inversa del Propagador Principal, al multiplicar a izquiera y a derecha por las matrices \ref{hslash_pq} y \ref{hslash_qp},
solo será distinto de cero el elemento $i'\rightarrow a', j \rightarrow b'$ del Propagador Principal. 

\begin{eqnarray}
	\begin{pmatrix}
		(h_{i',a'} | h_{i',a'}) & 0 & \cdots \\
		0 & \vdots & \vdots \\
		\vdots & \vdots & \vdots 
	\end{pmatrix} \otimes \begin{pmatrix}
		M_{i'a',i'a'} & M_{i'a',j' b'} &  \cdots \\
		M_{j' b',i'a'} & M_{j' b',j' b'} & \vdots \\
		\vdots & \vdots & \vdots
	\end{pmatrix} \otimes
	\begin{pmatrix}
		(h_{i',a'} | h_{i',a'}) & 0 & \cdots \\
		0 & \vdots & \vdots \\
		\vdots & \vdots & \vdots 
	\end{pmatrix} = M_{i'a',i'a'}
	\end{eqnarray}

Por lo que, al hacer una sumatoria sobre todos los orbitales $p,q$ diferentes de $i,j$ y $a,b$, obtendremos

\begin{equation}
	Tr\frac{\bm{h}_B| E \hat{I} - \hat{H}_0 | \bm{\widetilde{h}}_B }{Z_{B}} =  \sum_{p,q} \frac{M_{pq,pq}}{Z_{B}}
\end{equation}

\subsection*{Denominador}

Como se puede observar en el artículo \cite{Millan}, $Z$ representa la traza de $\rho$. Esto lo podemos observar
analizando la condición de la traza de una matriz densidad $\rho$ debe ser 1 

\begin{equation}
	Tr \rho =  \sum_k \frac{(\bm{h}_k|e^{| \bm{\widetilde{h}} ) (\bm{h}| E \hat{I} - \hat{H}_0 | \bm{\widetilde{h}} )   (\bm{h}|} |\bm{\widetilde{h}}_k) } 
	{Z} = 1
\end{equation}

Ya que la matriz densidad utilizando la teoría de Propagadores de Polarización \cite{QFT} está definida para todas las excitaciones del 
sistema molecular bajo estudio, generadas con el \textit{manifold} de operadores $\bm{h}$, y que se puede considerar
excitaciones de manera separada, de manera de poder escribir el espacio donde está definido el operador densidad 
como una composición de sub-sistemas  
\begin{equation}
	| \bm{\widetilde{h}}_{ia,jb} ) (\bm{h_{ia,jb}}| E \hat{I} - \hat{H}_0 | \bm{\widetilde{h}}_{ia,jb} )   (\bm{h_{ia,jb}}| \otimes
	| \bm{\widetilde{h}}_B ) (\bm{h_B}| E \hat{I} - \hat{H}_0 | \bm{\widetilde{h}}_B )   (\bm{h_B}|
\end{equation}

también podemos escribir $Z = Z_{ia,jb} \times Z_{B}$, por lo que la traza parcial de $\rho$ sobre el subsistema $B$ queda 

\begin{eqnarray}
	Tr_B \rho &=& \frac{e^{| {\bm{\widetilde{h}}_{ia,jb} ) (\bm{h}_{ia,jb}| E \hat{I} - \hat{H}_0 | 
	\bm{\widetilde{h}}_{ia,jb} )   (\bm{h}_{ia,jb}|}}}{Z_{ia,jb}} \otimes \frac{Tr(\rho_B)}{Z_{B}} \\
	&=& \rho_{ia,jb} 
\end{eqnarray}

\printbibliography

\end{document}
